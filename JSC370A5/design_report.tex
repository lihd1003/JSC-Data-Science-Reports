% Inherited from the classic.tplx https://pypi.org/project/nb-pdf-template/#description %
\documentclass[11pt]{article}

    \usepackage{parskip}
    \setcounter{secnumdepth}{0} %Suppress section numbers
    \usepackage[breakable]{tcolorbox}
    \tcbset{nobeforeafter}
    \usepackage{needspace}
    \usepackage{graphicx}
    \usepackage{caption}
    \usepackage{subcaption}

    \usepackage{iftex}
    \ifPDFTeX
    	\usepackage[T1]{fontenc}
    	\usepackage{mathpazo}
    \else
    	\usepackage{fontspec}
    \fi

    % Basic figure setup, for now with no caption control since it's done
    % automatically by Pandoc (which extracts ![](path) syntax from Markdown).
    \usepackage{graphicx}
    % Maintain compatibility with old templates. Remove in nbconvert 6.0
    \let\Oldincludegraphics\includegraphics
    % Ensure that by default, figures have no caption (until we provide a
    % proper Figure object with a Caption API and a way to capture that
    % in the conversion process - todo).
    \usepackage{caption}

    \usepackage[Export]{adjustbox} % Used to constrain images to a maximum size
    \adjustboxset{max size={0.9\linewidth}{0.9\paperheight}}
    \usepackage{float}
    \floatplacement{figure}{H} % forces figures to be placed at the correct location
    \usepackage{xcolor} % Allow colors to be defined
    \usepackage{enumerate} % Needed for markdown enumerations to work
    \usepackage{geometry} % Used to adjust the document margins
    \usepackage{amsmath} % Equations
    \usepackage{amssymb} % Equations
    \usepackage{textcomp} % defines textquotesingle
    % Hack from http://tex.stackexchange.com/a/47451/13684:
    \AtBeginDocument{%
        \def\PYZsq{\textquotesingle}% Upright quotes in Pygmentized code
    }
    \usepackage{upquote} % Upright quotes for verbatim code
    \usepackage{eurosym} % defines \euro
    \usepackage[mathletters]{ucs} % Extended unicode (utf-8) support
    \usepackage{fancyvrb} % verbatim replacement that allows latex
    \usepackage{grffile} % extends the file name processing of package graphics
                         % to support a larger range
    \makeatletter % fix for grffile with XeLaTeX
    \def\Gread@@xetex#1{%
      \IfFileExists{"\Gin@base".bb}%
      {\Gread@eps{\Gin@base.bb}}%
      {\Gread@@xetex@aux#1}%
    }
    \makeatother

    % The hyperref package gives us a pdf with properly built
    % internal navigation ('pdf bookmarks' for the table of contents,
    % internal cross-reference links, web links for URLs, etc.)
    \usepackage{hyperref}
    % The default LaTeX title has an obnoxious amount of whitespace. By default,
    % titling removes some of it. It also provides customization options.
    \usepackage{titling}
    \usepackage{longtable} % longtable support required by pandoc >1.10
    \usepackage{booktabs}  % table support for pandoc > 1.12.2
    \usepackage[inline]{enumitem} % IRkernel/repr support (it uses the enumerate* environment)
    \usepackage[normalem]{ulem} % ulem is needed to support strikethroughs (\sout)
                                % normalem makes italics be italics, not underlines
    \usepackage{mathrsfs}


    \let\Oldtex\TeX     % provide compatibility with nbconvert <= 5.3.1
    \let\Oldlatex\LaTeX % pre-included in nbconvert > 5.3.1 so redundant

    % Colors for the hyperref package
    \definecolor{urlcolor}{rgb}{0,.145,.698}
    \definecolor{linkcolor}{rgb}{.71,0.21,0.01}
    \definecolor{citecolor}{rgb}{.12,.54,.11}

    % ANSI colors
    \definecolor{ansi-black}{HTML}{3E424D}
    \definecolor{ansi-black-intense}{HTML}{282C36}
    \definecolor{ansi-red}{HTML}{E75C58}
    \definecolor{ansi-red-intense}{HTML}{B22B31}
    \definecolor{ansi-green}{HTML}{00A250}
    \definecolor{ansi-green-intense}{HTML}{007427}
    \definecolor{ansi-yellow}{HTML}{DDB62B}
    \definecolor{ansi-yellow-intense}{HTML}{B27D12}
    \definecolor{ansi-blue}{HTML}{208FFB}
    \definecolor{ansi-blue-intense}{HTML}{0065CA}
    \definecolor{ansi-magenta}{HTML}{D160C4}
    \definecolor{ansi-magenta-intense}{HTML}{A03196}
    \definecolor{ansi-cyan}{HTML}{60C6C8}
    \definecolor{ansi-cyan-intense}{HTML}{258F8F}
    \definecolor{ansi-white}{HTML}{C5C1B4}
    \definecolor{ansi-white-intense}{HTML}{A1A6B2}
    \definecolor{ansi-default-inverse-fg}{HTML}{FFFFFF}
    \definecolor{ansi-default-inverse-bg}{HTML}{000000}

    % commands and environments needed by pandoc snippets
    % extracted from the output of `pandoc -s`
    \providecommand{\tightlist}{%
      \setlength{\itemsep}{0pt}\setlength{\parskip}{0pt}}
    \DefineVerbatimEnvironment{Highlighting}{Verbatim}{commandchars=\\\{\}}
    % Add ',fontsize=\small' for more characters per line
    \newenvironment{Shaded}{}{}
    \newcommand{\KeywordTok}[1]{\textcolor[rgb]{0.00,0.44,0.13}{\textbf{{#1}}}}
    \newcommand{\DataTypeTok}[1]{\textcolor[rgb]{0.56,0.13,0.00}{{#1}}}
    \newcommand{\DecValTok}[1]{\textcolor[rgb]{0.25,0.63,0.44}{{#1}}}
    \newcommand{\BaseNTok}[1]{\textcolor[rgb]{0.25,0.63,0.44}{{#1}}}
    \newcommand{\FloatTok}[1]{\textcolor[rgb]{0.25,0.63,0.44}{{#1}}}
    \newcommand{\CharTok}[1]{\textcolor[rgb]{0.25,0.44,0.63}{{#1}}}
    \newcommand{\StringTok}[1]{\textcolor[rgb]{0.25,0.44,0.63}{{#1}}}
    \newcommand{\CommentTok}[1]{\textcolor[rgb]{0.38,0.63,0.69}{\textit{{#1}}}}
    \newcommand{\OtherTok}[1]{\textcolor[rgb]{0.00,0.44,0.13}{{#1}}}
    \newcommand{\AlertTok}[1]{\textcolor[rgb]{1.00,0.00,0.00}{\textbf{{#1}}}}
    \newcommand{\FunctionTok}[1]{\textcolor[rgb]{0.02,0.16,0.49}{{#1}}}
    \newcommand{\RegionMarkerTok}[1]{{#1}}
    \newcommand{\ErrorTok}[1]{\textcolor[rgb]{1.00,0.00,0.00}{\textbf{{#1}}}}
    \newcommand{\NormalTok}[1]{{#1}}

    % Additional commands for more recent versions of Pandoc
    \newcommand{\ConstantTok}[1]{\textcolor[rgb]{0.53,0.00,0.00}{{#1}}}
    \newcommand{\SpecialCharTok}[1]{\textcolor[rgb]{0.25,0.44,0.63}{{#1}}}
    \newcommand{\VerbatimStringTok}[1]{\textcolor[rgb]{0.25,0.44,0.63}{{#1}}}
    \newcommand{\SpecialStringTok}[1]{\textcolor[rgb]{0.73,0.40,0.53}{{#1}}}
    \newcommand{\ImportTok}[1]{{#1}}
    \newcommand{\DocumentationTok}[1]{\textcolor[rgb]{0.73,0.13,0.13}{\textit{{#1}}}}
    \newcommand{\AnnotationTok}[1]{\textcolor[rgb]{0.38,0.63,0.69}{\textbf{\textit{{#1}}}}}
    \newcommand{\CommentVarTok}[1]{\textcolor[rgb]{0.38,0.63,0.69}{\textbf{\textit{{#1}}}}}
    \newcommand{\VariableTok}[1]{\textcolor[rgb]{0.10,0.09,0.49}{{#1}}}
    \newcommand{\ControlFlowTok}[1]{\textcolor[rgb]{0.00,0.44,0.13}{\textbf{{#1}}}}
    \newcommand{\OperatorTok}[1]{\textcolor[rgb]{0.40,0.40,0.40}{{#1}}}
    \newcommand{\BuiltInTok}[1]{{#1}}
    \newcommand{\ExtensionTok}[1]{{#1}}
    \newcommand{\PreprocessorTok}[1]{\textcolor[rgb]{0.74,0.48,0.00}{{#1}}}
    \newcommand{\AttributeTok}[1]{\textcolor[rgb]{0.49,0.56,0.16}{{#1}}}
    \newcommand{\InformationTok}[1]{\textcolor[rgb]{0.38,0.63,0.69}{\textbf{\textit{{#1}}}}}
    \newcommand{\WarningTok}[1]{\textcolor[rgb]{0.38,0.63,0.69}{\textbf{\textit{{#1}}}}}


    % Define a nice break command that doesn't care if a line doesn't already
    % exist.
    \def\br{\hspace*{\fill} \\* }
    % Math Jax compatibility definitions
    \def\gt{>}
    \def\lt{<}
    \let\Oldtex\TeX
    \let\Oldlatex\LaTeX
    \renewcommand{\TeX}{\textrm{\Oldtex}}
    \renewcommand{\LaTeX}{\textrm{\Oldlatex}}
    % Document parameters
    % Document title




% Pygments definitions
\makeatletter
\def\PY@reset{\let\PY@it=\relax \let\PY@bf=\relax%
    \let\PY@ul=\relax \let\PY@tc=\relax%
    \let\PY@bc=\relax \let\PY@ff=\relax}
\def\PY@tok#1{\csname PY@tok@#1\endcsname}
\def\PY@toks#1+{\ifx\relax#1\empty\else%
    \PY@tok{#1}\expandafter\PY@toks\fi}
\def\PY@do#1{\PY@bc{\PY@tc{\PY@ul{%
    \PY@it{\PY@bf{\PY@ff{#1}}}}}}}
\def\PY#1#2{\PY@reset\PY@toks#1+\relax+\PY@do{#2}}

\expandafter\def\csname PY@tok@w\endcsname{\def\PY@tc##1{\textcolor[rgb]{0.73,0.73,0.73}{##1}}}
\expandafter\def\csname PY@tok@c\endcsname{\let\PY@it=\textit\def\PY@tc##1{\textcolor[rgb]{0.25,0.50,0.50}{##1}}}
\expandafter\def\csname PY@tok@cp\endcsname{\def\PY@tc##1{\textcolor[rgb]{0.74,0.48,0.00}{##1}}}
\expandafter\def\csname PY@tok@k\endcsname{\let\PY@bf=\textbf\def\PY@tc##1{\textcolor[rgb]{0.00,0.50,0.00}{##1}}}
\expandafter\def\csname PY@tok@kp\endcsname{\def\PY@tc##1{\textcolor[rgb]{0.00,0.50,0.00}{##1}}}
\expandafter\def\csname PY@tok@kt\endcsname{\def\PY@tc##1{\textcolor[rgb]{0.69,0.00,0.25}{##1}}}
\expandafter\def\csname PY@tok@o\endcsname{\def\PY@tc##1{\textcolor[rgb]{0.40,0.40,0.40}{##1}}}
\expandafter\def\csname PY@tok@ow\endcsname{\let\PY@bf=\textbf\def\PY@tc##1{\textcolor[rgb]{0.67,0.13,1.00}{##1}}}
\expandafter\def\csname PY@tok@nb\endcsname{\def\PY@tc##1{\textcolor[rgb]{0.00,0.50,0.00}{##1}}}
\expandafter\def\csname PY@tok@nf\endcsname{\def\PY@tc##1{\textcolor[rgb]{0.00,0.00,1.00}{##1}}}
\expandafter\def\csname PY@tok@nc\endcsname{\let\PY@bf=\textbf\def\PY@tc##1{\textcolor[rgb]{0.00,0.00,1.00}{##1}}}
\expandafter\def\csname PY@tok@nn\endcsname{\let\PY@bf=\textbf\def\PY@tc##1{\textcolor[rgb]{0.00,0.00,1.00}{##1}}}
\expandafter\def\csname PY@tok@ne\endcsname{\let\PY@bf=\textbf\def\PY@tc##1{\textcolor[rgb]{0.82,0.25,0.23}{##1}}}
\expandafter\def\csname PY@tok@nv\endcsname{\def\PY@tc##1{\textcolor[rgb]{0.10,0.09,0.49}{##1}}}
\expandafter\def\csname PY@tok@no\endcsname{\def\PY@tc##1{\textcolor[rgb]{0.53,0.00,0.00}{##1}}}
\expandafter\def\csname PY@tok@nl\endcsname{\def\PY@tc##1{\textcolor[rgb]{0.63,0.63,0.00}{##1}}}
\expandafter\def\csname PY@tok@ni\endcsname{\let\PY@bf=\textbf\def\PY@tc##1{\textcolor[rgb]{0.60,0.60,0.60}{##1}}}
\expandafter\def\csname PY@tok@na\endcsname{\def\PY@tc##1{\textcolor[rgb]{0.49,0.56,0.16}{##1}}}
\expandafter\def\csname PY@tok@nt\endcsname{\let\PY@bf=\textbf\def\PY@tc##1{\textcolor[rgb]{0.00,0.50,0.00}{##1}}}
\expandafter\def\csname PY@tok@nd\endcsname{\def\PY@tc##1{\textcolor[rgb]{0.67,0.13,1.00}{##1}}}
\expandafter\def\csname PY@tok@s\endcsname{\def\PY@tc##1{\textcolor[rgb]{0.73,0.13,0.13}{##1}}}
\expandafter\def\csname PY@tok@sd\endcsname{\let\PY@it=\textit\def\PY@tc##1{\textcolor[rgb]{0.73,0.13,0.13}{##1}}}
\expandafter\def\csname PY@tok@si\endcsname{\let\PY@bf=\textbf\def\PY@tc##1{\textcolor[rgb]{0.73,0.40,0.53}{##1}}}
\expandafter\def\csname PY@tok@se\endcsname{\let\PY@bf=\textbf\def\PY@tc##1{\textcolor[rgb]{0.73,0.40,0.13}{##1}}}
\expandafter\def\csname PY@tok@sr\endcsname{\def\PY@tc##1{\textcolor[rgb]{0.73,0.40,0.53}{##1}}}
\expandafter\def\csname PY@tok@ss\endcsname{\def\PY@tc##1{\textcolor[rgb]{0.10,0.09,0.49}{##1}}}
\expandafter\def\csname PY@tok@sx\endcsname{\def\PY@tc##1{\textcolor[rgb]{0.00,0.50,0.00}{##1}}}
\expandafter\def\csname PY@tok@m\endcsname{\def\PY@tc##1{\textcolor[rgb]{0.40,0.40,0.40}{##1}}}
\expandafter\def\csname PY@tok@gh\endcsname{\let\PY@bf=\textbf\def\PY@tc##1{\textcolor[rgb]{0.00,0.00,0.50}{##1}}}
\expandafter\def\csname PY@tok@gu\endcsname{\let\PY@bf=\textbf\def\PY@tc##1{\textcolor[rgb]{0.50,0.00,0.50}{##1}}}
\expandafter\def\csname PY@tok@gd\endcsname{\def\PY@tc##1{\textcolor[rgb]{0.63,0.00,0.00}{##1}}}
\expandafter\def\csname PY@tok@gi\endcsname{\def\PY@tc##1{\textcolor[rgb]{0.00,0.63,0.00}{##1}}}
\expandafter\def\csname PY@tok@gr\endcsname{\def\PY@tc##1{\textcolor[rgb]{1.00,0.00,0.00}{##1}}}
\expandafter\def\csname PY@tok@ge\endcsname{\let\PY@it=\textit}
\expandafter\def\csname PY@tok@gs\endcsname{\let\PY@bf=\textbf}
\expandafter\def\csname PY@tok@gp\endcsname{\let\PY@bf=\textbf\def\PY@tc##1{\textcolor[rgb]{0.00,0.00,0.50}{##1}}}
\expandafter\def\csname PY@tok@go\endcsname{\def\PY@tc##1{\textcolor[rgb]{0.53,0.53,0.53}{##1}}}
\expandafter\def\csname PY@tok@gt\endcsname{\def\PY@tc##1{\textcolor[rgb]{0.00,0.27,0.87}{##1}}}
\expandafter\def\csname PY@tok@err\endcsname{\def\PY@bc##1{\setlength{\fboxsep}{0pt}\fcolorbox[rgb]{1.00,0.00,0.00}{1,1,1}{\strut ##1}}}
\expandafter\def\csname PY@tok@kc\endcsname{\let\PY@bf=\textbf\def\PY@tc##1{\textcolor[rgb]{0.00,0.50,0.00}{##1}}}
\expandafter\def\csname PY@tok@kd\endcsname{\let\PY@bf=\textbf\def\PY@tc##1{\textcolor[rgb]{0.00,0.50,0.00}{##1}}}
\expandafter\def\csname PY@tok@kn\endcsname{\let\PY@bf=\textbf\def\PY@tc##1{\textcolor[rgb]{0.00,0.50,0.00}{##1}}}
\expandafter\def\csname PY@tok@kr\endcsname{\let\PY@bf=\textbf\def\PY@tc##1{\textcolor[rgb]{0.00,0.50,0.00}{##1}}}
\expandafter\def\csname PY@tok@bp\endcsname{\def\PY@tc##1{\textcolor[rgb]{0.00,0.50,0.00}{##1}}}
\expandafter\def\csname PY@tok@fm\endcsname{\def\PY@tc##1{\textcolor[rgb]{0.00,0.00,1.00}{##1}}}
\expandafter\def\csname PY@tok@vc\endcsname{\def\PY@tc##1{\textcolor[rgb]{0.10,0.09,0.49}{##1}}}
\expandafter\def\csname PY@tok@vg\endcsname{\def\PY@tc##1{\textcolor[rgb]{0.10,0.09,0.49}{##1}}}
\expandafter\def\csname PY@tok@vi\endcsname{\def\PY@tc##1{\textcolor[rgb]{0.10,0.09,0.49}{##1}}}
\expandafter\def\csname PY@tok@vm\endcsname{\def\PY@tc##1{\textcolor[rgb]{0.10,0.09,0.49}{##1}}}
\expandafter\def\csname PY@tok@sa\endcsname{\def\PY@tc##1{\textcolor[rgb]{0.73,0.13,0.13}{##1}}}
\expandafter\def\csname PY@tok@sb\endcsname{\def\PY@tc##1{\textcolor[rgb]{0.73,0.13,0.13}{##1}}}
\expandafter\def\csname PY@tok@sc\endcsname{\def\PY@tc##1{\textcolor[rgb]{0.73,0.13,0.13}{##1}}}
\expandafter\def\csname PY@tok@dl\endcsname{\def\PY@tc##1{\textcolor[rgb]{0.73,0.13,0.13}{##1}}}
\expandafter\def\csname PY@tok@s2\endcsname{\def\PY@tc##1{\textcolor[rgb]{0.73,0.13,0.13}{##1}}}
\expandafter\def\csname PY@tok@sh\endcsname{\def\PY@tc##1{\textcolor[rgb]{0.73,0.13,0.13}{##1}}}
\expandafter\def\csname PY@tok@s1\endcsname{\def\PY@tc##1{\textcolor[rgb]{0.73,0.13,0.13}{##1}}}
\expandafter\def\csname PY@tok@mb\endcsname{\def\PY@tc##1{\textcolor[rgb]{0.40,0.40,0.40}{##1}}}
\expandafter\def\csname PY@tok@mf\endcsname{\def\PY@tc##1{\textcolor[rgb]{0.40,0.40,0.40}{##1}}}
\expandafter\def\csname PY@tok@mh\endcsname{\def\PY@tc##1{\textcolor[rgb]{0.40,0.40,0.40}{##1}}}
\expandafter\def\csname PY@tok@mi\endcsname{\def\PY@tc##1{\textcolor[rgb]{0.40,0.40,0.40}{##1}}}
\expandafter\def\csname PY@tok@il\endcsname{\def\PY@tc##1{\textcolor[rgb]{0.40,0.40,0.40}{##1}}}
\expandafter\def\csname PY@tok@mo\endcsname{\def\PY@tc##1{\textcolor[rgb]{0.40,0.40,0.40}{##1}}}
\expandafter\def\csname PY@tok@ch\endcsname{\let\PY@it=\textit\def\PY@tc##1{\textcolor[rgb]{0.25,0.50,0.50}{##1}}}
\expandafter\def\csname PY@tok@cm\endcsname{\let\PY@it=\textit\def\PY@tc##1{\textcolor[rgb]{0.25,0.50,0.50}{##1}}}
\expandafter\def\csname PY@tok@cpf\endcsname{\let\PY@it=\textit\def\PY@tc##1{\textcolor[rgb]{0.25,0.50,0.50}{##1}}}
\expandafter\def\csname PY@tok@c1\endcsname{\let\PY@it=\textit\def\PY@tc##1{\textcolor[rgb]{0.25,0.50,0.50}{##1}}}
\expandafter\def\csname PY@tok@cs\endcsname{\let\PY@it=\textit\def\PY@tc##1{\textcolor[rgb]{0.25,0.50,0.50}{##1}}}

\def\PYZbs{\char`\\}
\def\PYZus{\char`\_}
\def\PYZob{\char`\{}
\def\PYZcb{\char`\}}
\def\PYZca{\char`\^}
\def\PYZam{\char`\&}
\def\PYZlt{\char`\<}
\def\PYZgt{\char`\>}
\def\PYZsh{\char`\#}
\def\PYZpc{\char`\%}
\def\PYZdl{\char`\$}
\def\PYZhy{\char`\-}
\def\PYZsq{\char`\'}
\def\PYZdq{\char`\"}
\def\PYZti{\char`\~}
% for compatibility with earlier versions
\def\PYZat{@}
\def\PYZlb{[}
\def\PYZrb{]}
\makeatother

    %Reconfigured pygments
    \makeatletter
    \expandafter\def\csname PY@tok@mi\endcsname{\def\PY@tc##1{\textcolor[HTML]{008800}{##1}}} %numbers
    \expandafter\def\csname PY@tok@mf\endcsname{\def\PY@tc##1{\textcolor[HTML]{008800}{##1}}} %numbers
    \expandafter\def\csname PY@tok@nn\endcsname{\def\PY@tc##1{\textcolor[HTML]{000000}{##1}}} %imports
    \expandafter\def\csname PY@tok@ow\endcsname{\let\PY@bf=\textbf\def\PY@tc##1{\textcolor[HTML]{008000}{##1}}} %operator.word
    \expandafter\def\csname PY@tok@o\endcsname{\def\PY@tc##1{\textcolor[HTML]{AA22FF}{\codetrue##1\codefalse}}} %operator
    \makeatother

    \makeatletter
    \newcommand*\@iflatexlater{\@ifl@t@r\fmtversion}
    \@iflatexlater{2016/03/01}{
	    \newcommand{\wordboundary}{4095}}{
	    \newcommand{\wordboundary}{255}}
    \makeatother

    \newif\ifcode
    \codefalse
    \definecolor{Grey}{rgb}{0.40,0.40,0.40}
    %If using XeLaTeX, use magic to not highlight . operators with purple.
    \ifdefined\XeTeXcharclass
    \XeTeXinterchartokenstate = 1
    \newXeTeXintercharclass \mycharclassGrey
    \XeTeXcharclass `. \mycharclassGrey
    \XeTeXinterchartoks 0 \mycharclassGrey   = {\bgroup\ifcode\color{Grey}\else\fi}

    \XeTeXinterchartoks \wordboundary \mycharclassGrey = {\bgroup\ifcode\color{Grey}\else\fi}

    \XeTeXinterchartoks \mycharclassGrey 0   = {\egroup}
    \XeTeXinterchartoks \mycharclassGrey \wordboundary = {\egroup}
    \fi %end magical operator highlighting
    %End Reconfigured Pygments


    % Exact colors from NB
    \definecolor{incolor}{HTML}{303F9F}
    \definecolor{outcolor}{HTML}{D84315}
    \definecolor{cellborder}{HTML}{CFCFCF}
    \definecolor{cellbackground}{HTML}{F7F7F7}

    % needed definitions
    \newif\ifleftmargins
    \newlength{\promptlength}

    % cell style settings
        \leftmarginsfalse


    % prompt
    \newcommand{\prompt}[3]{
        \needspace{1.1cm}
        \settowidth{\promptlength}{ #1 [#3] }
        \ifleftmargins\hspace{-\promptlength}\hspace{-5pt}\fi
        {\color{#2}#1 [#3]:}
        \ifleftmargins\vspace{-2.7ex}\fi
    }


    % environments
    \newenvironment{InVerbatim}{\VerbatimEnvironment%
        \begin{tcolorbox}[breakable, size=fbox, boxrule=1pt, pad at break*=1mm,
            colback=cellbackground, colframe=cellborder]
            \begin{Verbatim}
            }{
            \end{Verbatim}
        \end{tcolorbox}
    }
    \newenvironment{OutVerbatim}{\VerbatimEnvironment%
        \begin{tcolorbox}[breakable, boxrule=.5pt, size=fbox, pad at break*=1mm, opacityfill=0]
            \begin{Verbatim}
            }{
            \end{Verbatim}
        \end{tcolorbox}
    }

    %Updated MathJax Compatibility (if future behaviour of the notebook changes this may be removed)
    \renewcommand{\TeX}{\ifmmode \textrm{\Oldtex} \else \textbackslash TeX \fi}
    \renewcommand{\LaTeX}{\ifmmode \Oldlatex \else \textbackslash LaTeX \fi}

    % Header Adjustments
    \renewcommand{\paragraph}{\textbf}
    \renewcommand{\subparagraph}[1]{\textit{\textbf{#1}}}


    % Prevent overflowing lines due to hard-to-break entities
    \sloppy
    % Setup hyperref package
    \hypersetup{
      breaklinks=true,  % so long urls are correctly broken across lines
      colorlinks=true,
      urlcolor=urlcolor,
      linkcolor=linkcolor,
      citecolor=citecolor,
      }
    % Slightly bigger margins than the latex defaults
    \geometry{tmargin=.5in,bmargin=.7in,lmargin=.7in,rmargin=.5in}


\begin{document}

    \title{Web Development is Redundantly Complicated}\author{Haoda Li}


\date{\today}
\maketitle







    \hypertarget{summary}{%
\section{Summary}\label{summary}}

\emph{The experience with JavaScript, the primary programming language
for front end development, is not pleasant. There are too many choices
doing the same thing, while they have their own implementations that are
not compatible with each other. Further more, the many frameworks vary
so much in syntax, while similar in functionality. I've also seen
similar voice on Quora, Reddit, and Medium.}

\emph{This poster is targeted at the web developers. I want to find
clues to show that the WebDev community is suffering from these
unnecessary redundancy. Therefore, the developers can think about how to
improve the community by building better standards and stop writing
inefficient code.}

    \hypertarget{part-i.-a-glance-of-the-data}{%
\section{1. A Glance of the
Data}\label{part-i.-a-glance-of-the-data}}

The data is found from Libraries.io, who tracks 37 package managers and
3 open source repository platforms. Near 5 million projects and
repositories are collected, and their dependencies were tracked between
2015 and 2018.

This dataset \ref{fig:data} is quite sophisticated, it includes the primary language of
the project, the number of dependent projects, as well as the keywords
and description. These information is extremely useful for my purpose.
As most web projects are developed in primarily JavaScript and its
derivatives, I can filter out web projects by the language. Also, I can
use the keywords and description to determine the general purpose of the
project.

On the other side, not every project will grow to be a good project.
Most of the projects are in low-quality and/or abandoned by the project
owner. Libraries.io proposes one metric, the SourceRank, to measure the
quality of the project. SourceRank is primarily determined by the number
of dependent projects, the last update date, and the quality of the
documentation. I dropped all projects that is lower or equal to 5
SourceRank score, which are mostly sketchy projects without any
documentation and not updated for quite a long time.



\begin{figure}[h!]
  \centering
  \begin{subfigure}[b]{0.6\linewidth}
    \includegraphics[width=\linewidth]{./plots/intro.png}
  \end{subfigure}
  \begin{subfigure}[b]{0.3\linewidth}
    \includegraphics[width=\linewidth]{./plots/intro_dep.png}
  \end{subfigure}
  \caption{A Glance of the datasets}
  \label{fig:data}
\end{figure}



    \hypertarget{think-about-the-story}{%
\section{2. Think About the Story}\label{think-about-the-story}}

Looking at the huge data, I spent a long time determining what should be
put on a poster. Considering the audience, who should be developers or
the public without deep understanding with statistics. I decided to make
the story visually appealing and simple in concepts.

\hypertarget{webdev-is-complicated}{%
\subsection{2.1 WebDev is Complicated}\label{webdev-is-complicated}}

The thought is very natural, to show the size of WebDev community and to
show the complexity of working on these projects.

\hypertarget{web-world-is-huge}{%
\subsubsection{2.1.1 Web world is Huge}\label{web-world-is-huge}}

The first thing to show is that JavaScript dominates in number of
projects. Although developers know that JavaScript has the largest
community among all languages, most people may not understand how
dominant it is. I want my audience to have a very direct view on the
size.

\hypertarget{web-world-is-messy}{%
\subsubsection{2.1.2. Web world is Messy}\label{web-world-is-messy}}

I proposed several measures for ``complexity'' of a language. For
example, I tried the depth of the dependency tree. However, I found most
of the measurements are not easy to understand or it is not significant
enough for the comparisons between JavaScript and other languages. I
finally choose to use the average number of dependencies. This
measurement is simple and JavaScript clearly beats others.

\hypertarget{redundancy}{%
\subsection{2.2 Redundancy}\label{redundancy}}

This is a very hard measurement. The initial idea is to build community
graph from dependencies and then measure how similar their distribution
over keywords are. However, I quickly faced many issue with this
approach. First, the size of the dataset is still too large to construct
a the dependency graph. Second, the concept is too complex to explain
these in a poster.

\hypertarget{redundancy-in-simple-tasks}{%
\subsubsection{2.2.1. Redundancy in Simple
Tasks}\label{redundancy-in-simple-tasks}}

I worked around by ``attacking'' JavaScript's own historical problem.
Because JavaScript is not a well-designed language, it lacks a lot of
standard libraries for simple tasks, hence the community build packages
fill in. However, due to the lack of communication, many community
packages work on the same problem and are incompatible with each other.
Such problems have troubled me a lot, and I've seen many memes on the
issues. Therefore, I used the same approach by showing how JavaScript
have redundant packages on the simple tasks.

\hypertarget{redundancy-in-frameworks}{%
\subsubsection{2.2.2. Redundancy in
Frameworks}\label{redundancy-in-frameworks}}

Finally, I looked at the redundancy over frameworks. There are so many
question on the Internet about which framework should web developers
choose, while they don't show much difference over these packages. I
decided to leave this topic as the final thoughts. These frameworks are
leading ways for nowadays web development. However, do the webdev
community need these many of frameworks on the same thing.

    \hypertarget{designing-process}{%
\section{3. Designing Process}\label{designing-process}}

\hypertarget{the-layout}{%
\subsection{3.1 The Layout}\label{the-layout}}

The design is more for the flow of the story, and the actual content I
want to fit. The first design (left \ref{fig:layout}) put more weights on the
``redundancy'' part, while I realized that the first part is too crowded
and the two parts should be equally important. In addition, the four
parts grid fits my story better.


\begin{figure}
  \centering
  \includegraphics[width=0.9\textwidth]{design_report_files/design_report_8_0.jpg}
  \caption{The layout}
  \label{fig:layout}
\end{figure}

    \hypertarget{the-plots}{%
\subsection{3.2 The Plots}\label{the-plots}}

    Many decisions are made on the plots. For showing the WebDev world is
huge. I tried to use a bar plot and a pie chart to show the number of
projects in JavaScript and the market share it has. However, I find that
the two plots have the same information. Therefore, I tried to plot the
number of new projects in each language through time as a stacked bar
plot, while it has too much information and does not convey the message
well. Finally, I choose to have the line plot of the cumulative number
of projects for each language, along with the market share as a bar.

For the redundancy part, I tried to balance between the amount of data
manipulation and the intuition. I've thought about using the
distribution of keywords. However, the information is not quite clear
for the audience, and overall looks messy. Finally, I focused more on
the intuition side. The plots I used are not ``friendly'' for
statisticians and they don't have quite the exact information (circles,
colors). However, they are more visually appealing for the public and
allow the audience to consider more on the message.

\begin{figure}[h!]
  \centering
  \begin{subfigure}[b]{0.45\linewidth}
    \includegraphics[width=\linewidth]{design_report_files/design_report_12_0.png}
  \end{subfigure}
  \begin{subfigure}[b]{0.45\linewidth}
    \includegraphics[width=\linewidth]{design_report_files/design_report_13_0.png}
  \end{subfigure}
  \begin{subfigure}[b]{0.45\linewidth}
    \includegraphics[width=\linewidth]{design_report_files/design_report_14_0.png}
  \end{subfigure}
  \caption{Design of the plots}
  \label{fig:plots}
\end{figure}


    \hypertarget{twinkling-the-details}{%
\subsection{3.3 Twinkling the Details}\label{twinkling-the-details}}

Finally, I can knit the figures into a story. For example, I slightly
shift the position of the introduction and add a word cloud on the top
right. The word cloud is a ``glance'' of the WebDev world and the
developers can be attracted by it. Also, I unify the color and share the
legend for some of the figures to make it look nice and precise.
Finally, I chose green for the overall color scheme, green is the color
of Github's contribution plot. The color hints the open source
development community and their contributions to the projects.

    \hypertarget{reflections-and-conclusions}{%
\section{4. Reflections and
Conclusions}\label{reflections-and-conclusions}}

First, the most significant difference for a poster is the amount of
explanations and information it can hold. Compare to a data report or a
interactive dashboard, a poster contains significantly less information.
Most of the information are carried through the figures and the lack of
interactivity further limit the amount of information. Therefore, we
have to focus on the key ideas and highlight it by size, color, or
shape. Such task is quite challenging.

Also, the targeted audience matters. Because the difference in the
audience and the level of background knowledge. A poster will focus less
on the technical details and focus more on the appealing visuals.
Sometimes, these visuals may not be very statistically useful, but
attractive for the audience.

In addition, Compare to a data analysis or data explorations, the poster
is more biased on nature. When I try to convey our audience with the
story, I chose clues that are on my side. We choose the data that looks
appealing and drop some data that are important but against our story.
Although every data analysis will have some degree of bias, the poster
is more impacted by these bias due to the targeted audience and limited
information it contains.
















    % Add a bibliography block to the postdoc



\end{document}
